%%%%%%%%%%%%%%%%%%%%%%%%%%%%%%%%%%%%%%%%%%%%%%%%%%%%%%%%
%%%%%%%%%%%%%%%%%%%%%%%%%%%%%%%%%%%%%%%%%%%%%%%%%%%%%%%%
%%%                                                  %%%
%%% University of Arizona themed Beamer presentation %%%
%%% Based on the Warsaw, Palo Alto and UNL templates %%%
%%% modified by Joseph V. Casillas (6-13-2011)       %%%
%%%                                                  %%%
%%%%%%%%%%%%%%%%%%%%%%%%%%%%%%%%%%%%%%%%%%%%%%%%%%%%%%%%
%%%%%%%%%%%%%%%%%%%%%%%%%%%%%%%%%%%%%%%%%%%%%%%%%%%%%%%%

\documentclass{beamer}
\mode<presentation>{\usetheme{UA3}\setbeamercovered{transparent}}
\usepackage[english]{babel}
\usepackage[utf8]{inputenc}
\usepackage{t1enc}

\usepackage{verbatim}
\usepackage{media9}
\usepackage{apacite}
\usepackage{tipa}
\usepackage{amssymb}   
\usepackage{soul}               % \hl{highlight this text}
\usepackage{graphicx}           % Put .eps and .pdf images into document
\usepackage{colortbl}
\usepackage{times}
\usepackage{url}   				%this allows us to cite URLs in the text
\usepackage{qtree}

\title{UA themed Beamer presentation}

\author[LastName]{FirstName LastName}

\institute{University of Arizona}

\date{\today}

\subject{Linguistics}

\begin{document}

%%%%%%%%%%%%%%%%%%%%%%%%%%%%%%%%%
%%%%%%%%%%%%%%%%%%%%%%%%%%%%%%%%%
\begin{frame}%Página principal
  \titlepage
\end{frame}

\begin{frame}
	\tableofcontents
\end{frame}

\section{Overview} % (fold)
\label{sec:overview}

\begin{frame}{Overview}
	To change the style of the presentation change the ``usetheme'' option en the preamble. \pause
	\begin{itemize}
		\item There are 3 options: \pause
		\begin{itemize}
			\item UA - Lateral column on the left side (red and blue) 
			\item UA2 - Bars on the top and bottom (red and blue) 
			\item UA3 - Bars on the top and bottom (grey, more formal) 
		\end{itemize} \pause
		\item You may have to compile the tex file two or three times for all the sections to show up.
	\end{itemize}
\end{frame}
% section overview (end)


\section{Installation} % (fold)
\label{sec:installation}

\begin{frame}{Installation}
	\begin{itemize}
		\item You can use the UA theme as long as the .sty files are in the same folder as your presentation.
		\item If you would like to permanently use the theme, copy the .sty folders into your \LaTeX\ ``texmf'' folder.
	\end{itemize}
	On my computer it looks like this: \\
	\vspace{2.5mm}
	/Users/{your\_user\_name}/Library/texmf/tex/latex/misc
\end{frame}
% section installation (end)


\section{Examples} % (fold)
\label{sec:examples}

\begin{frame}{Examples}
	\begin{block}{Block example}
		\begin{description}
			\item[Label] description \pause
			\item[Label] description \pause
			\item[Label] description
		\end{description}
	\end{block}
\end{frame}

\begin{frame}{Asimlación de las nasales}
	\begin{center}
		{\scriptsize
		\renewcommand{\arraystretch}{1.5}
		\begin{tabular}{@{}|l|l|l|l|l|@{}}
			\hline
			Fonema & Alófono & Punto & Ejemplo & \\ 
			\hline
			/m/ inicial & [m] + vocal                 & Bilabial       &  Mamá, cama     & \\ 
			/n/ inicial & [n] + vocal                 & Alveolar       &  Carne, nada    & \\ 
			/ñ/ inicial & [\textltailn] + vocal       & Palatal        &  Caña, ñoña     & \\ 
			\hline                                                               
			            & [m] + /p,b/                 & bilabial       &  “un barco”     & [ú{\bf m.b}ár.ko] \\ 
			            & [\textltailm] + /f/         & labiodental    &  “en Finlandia” & [e{\bf \textltailm.f}in.lá\textsubbridge{n}.d\textsubarch{i}a] \\ 
			/N/ final   & [\textsubbridge{n}] + /t,d/ & dental         &  “en Finlandia” & [e\textltailm.fin.lá{\bf \textsubbridge{n}.d}\textsubarch{i}a] \\ 
			de sílaba   & [n] + /s,l,r/               & alveolar       &  “ansiedad”     & [a{\bf n.s}\textsubarch{i}e.\dh á\dh] \\ 
			            & [\'n] + /\textteshlig/      & alveopalatal   &  “ancho”        & [á{\bf \'n.\textteshlig} o] \\ 
			            & [\textltailn] + /j/         & palatal        &  “inyección”    & [i{\bf \textltailn.j} ek.s\textsubarch{i}ón] \\ 
			            & [\textipa{N}] + /k,g,x/     & velar          &  “tengo”        & [té{\bf \textipa{N}.g}o] \\ 
			\hline
		\end{tabular}}
	\end{center}
\end{frame}
% section examples (end)

\end{document}